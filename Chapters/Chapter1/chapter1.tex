%!TEX root = ../thesis.tex
%*******************************************************************************
%*********************************** First Chapter *****************************
%*******************************************************************************

\chapter{Introduction}  %Title of the First Chapter

\ifpdf
    \graphicspath{{Chapters/Chapter1/Figs/Raster/}{Chapters/Chapter1/Figs/PDF/}{Chapters/Chapter1/Figs/}}
\else
    \graphicspath{{Chapters/Chapter1/Figs/Vector/}{Chapters/Chapter1/Figs/}}
\fi


%********************************** %First Section  **************************************
\section{The honeybee and the division of labor} %Section - 1.1 %TODO: is the title good?
%TODO: indent the first paragraph 
A honeybee (Apis mellifera) colony contains multiple fascinating examples of complex adaptive behaviour. 
Localized cues exchanged between individuals amount to emergent directional signals for the entire colony 
in ways heavily investigated, but often still not completely understood. 

One of the most notable and well-researched adaptive mechanisms of a colony is its division of labor (DOL). 
During the winter (a season of low activity for the bees), the colony focuses on survival 
and its workers are generalists, performing sets of tasks not easily distinguishable from those of other workers. 
For the spring-summer season, however, the hive’s goals change and along with them, the patterns of labor division. 
Hive growth and resource accumulation take priority, and specialization eventuates amongst workers. 
They begin to fill different roles, the allocation of which highly correlates with age (an effect 
known as temporal polyethism), but is also grounded in the colony’s current needs and in environmental 
factors affecting it (adaptive behaviour) \citep{Aup91} (Seeley 1982, Johnson 2008). Groups of workers that can be 
categorized as performing the same set of tasks are commonly referred to as castes. It is common to recognize 
four of them in the temporal caste system that the worker bees exhibit in the summer: 
cell cleaners, nurses, middle-aged bees (MABs), and foragers. 
This work concentrates on the transition between MABs and foragers, 
possibly the most distinguishable and important in the lifecycle of a bee.


%********************************** %Second Section  **************************************
\section{The foraging phase}
The foraging phase is the last one in a bee’s life and it comes with an increased risk of death. 
It is often proposed that this has to do with the extreme strain foraging puts on their bodies - 
essentially causing them to work themselves to death. This is supported for example by (Williams et al, 2008), 
who have shown honeybee flight to cause extremely high metabolic rates and induce oxidative stress, 
likely significantly accelerating the ageing process and causing early deaths. On the other hand, 
the results of \citep{Aup91} (Visschard and Dukes, 1997) suggest that foragers’ deaths are usually caused not by senescence, 
but by the heightened risks of outside life (like predation) that come with their function. 
According to their data, mortality rates were constant with respect to age, and not accelerating - 
as would have been suggested by the body strain hypothesis. 

Regardless of the reasons behind it, foraging nearly always ends in the death of a bee. 
That fact, combined with an estimate of the length of the foraging phase 
(mean of 7.7 days ± 0.75 days, median of 7 days, and range of 2 to 17 day according to \citep{Aup91} Visschard and Dukes, 1997), 
should allow us to get a very simple estimate of the foraging period, which we can then use to validate the results 
we produce with more involved methods.
